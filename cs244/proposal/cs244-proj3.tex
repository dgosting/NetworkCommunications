\documentclass[letter paper, 11pt]{article}
\usepackage{amsmath, enumerate, amssymb, amsthm, graphicx}
\usepackage[left = 1in, right = 1in, top = 1in]{geometry}

\title{CS 244 Project 3}
\author{Sweet Song(shihui) and Jason Zhao(jlzhao)}
\date{}

\begin{document}
\maketitle{}
\section*{Jellyfish - Networking Datacenter Randomly}
\subsection*{Synopsis}
This paper proposes a new network topology for data centers. Jellyfish uses a random graph to create a high bandwidth and fault tolerant network for data centers compared to traditional methods such at fat trees. The main idea behind random topologies is that it has shorter paths between servers and also have multiple disjoint paths to increase bandwidth. The authors noticed that existing path finding algorithms such as ECMP is not suffient to reach maximum capacity and propose several new path finding algorithms to increase perform. In the end, the authors show that random graph topologies has similar bisection throughput compared to traditional fat trees and is more extensible when adding new servers.

\subsection*{Results to Reproduce}
The most interesting aspect of the paper is showing that a datacenter running using random graph topology can achieve the same actual bisection bandwidth as traditional fat tree topologies. We plan on replicating the results in figure 9 and figure 11 and hope to show that Jellyfish scales with fat tree as the number of servers in the data center increase. The authors used simulators to compute the expected throughput and we plan on using Mininet to achieve the same result. In addition to replicating the authors results, we aim to examine if there are any improvements to the path finding algorithm used by the authors. The conclusions from the authors were to use k-shortest path with Multi-Path TCP, we want to see if there is any other protocol or algorithm that will leverage the random topology more efficiently.

\subsection*{Testing Methodology}
We will run our experiment on datacenters containing between 1 and 5000 servers at some fixed interval. We will use the authors random sampling algorithm to generate the topology and use Mininet to emulate the data center. In addition, we will implement our own switch logic and end host logic to support multi-path tcp and k-shortest path routing. We will also implement a fat-tree topology using the same number of servers and compare the aggregate througput of the 2 networks. We will test both topologies using both a uniform workload and data shuffle workload similar to traffic seen in Map Reduce applications. 
\end{document}
